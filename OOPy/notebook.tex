
% Default to the notebook output style

    


% Inherit from the specified cell style.




    
\documentclass[11pt]{article}

    
    
    \usepackage[T1]{fontenc}
    % Nicer default font (+ math font) than Computer Modern for most use cases
    \usepackage{mathpazo}

    % Basic figure setup, for now with no caption control since it's done
    % automatically by Pandoc (which extracts ![](path) syntax from Markdown).
    \usepackage{graphicx}
    % We will generate all images so they have a width \maxwidth. This means
    % that they will get their normal width if they fit onto the page, but
    % are scaled down if they would overflow the margins.
    \makeatletter
    \def\maxwidth{\ifdim\Gin@nat@width>\linewidth\linewidth
    \else\Gin@nat@width\fi}
    \makeatother
    \let\Oldincludegraphics\includegraphics
    % Set max figure width to be 80% of text width, for now hardcoded.
    \renewcommand{\includegraphics}[1]{\Oldincludegraphics[width=.8\maxwidth]{#1}}
    % Ensure that by default, figures have no caption (until we provide a
    % proper Figure object with a Caption API and a way to capture that
    % in the conversion process - todo).
    \usepackage{caption}
    \DeclareCaptionLabelFormat{nolabel}{}
    \captionsetup{labelformat=nolabel}

    \usepackage{adjustbox} % Used to constrain images to a maximum size 
    \usepackage{xcolor} % Allow colors to be defined
    \usepackage{enumerate} % Needed for markdown enumerations to work
    \usepackage{geometry} % Used to adjust the document margins
    \usepackage{amsmath} % Equations
    \usepackage{amssymb} % Equations
    \usepackage{textcomp} % defines textquotesingle
    % Hack from http://tex.stackexchange.com/a/47451/13684:
    \AtBeginDocument{%
        \def\PYZsq{\textquotesingle}% Upright quotes in Pygmentized code
    }
    \usepackage{upquote} % Upright quotes for verbatim code
    \usepackage{eurosym} % defines \euro
    \usepackage[mathletters]{ucs} % Extended unicode (utf-8) support
    \usepackage[utf8x]{inputenc} % Allow utf-8 characters in the tex document
    \usepackage{fancyvrb} % verbatim replacement that allows latex
    \usepackage{grffile} % extends the file name processing of package graphics 
                         % to support a larger range 
    % The hyperref package gives us a pdf with properly built
    % internal navigation ('pdf bookmarks' for the table of contents,
    % internal cross-reference links, web links for URLs, etc.)
    \usepackage{hyperref}
    \usepackage{longtable} % longtable support required by pandoc >1.10
    \usepackage{booktabs}  % table support for pandoc > 1.12.2
    \usepackage[inline]{enumitem} % IRkernel/repr support (it uses the enumerate* environment)
    \usepackage[normalem]{ulem} % ulem is needed to support strikethroughs (\sout)
                                % normalem makes italics be italics, not underlines
    

    
    
    % Colors for the hyperref package
    \definecolor{urlcolor}{rgb}{0,.145,.698}
    \definecolor{linkcolor}{rgb}{.71,0.21,0.01}
    \definecolor{citecolor}{rgb}{.12,.54,.11}

    % ANSI colors
    \definecolor{ansi-black}{HTML}{3E424D}
    \definecolor{ansi-black-intense}{HTML}{282C36}
    \definecolor{ansi-red}{HTML}{E75C58}
    \definecolor{ansi-red-intense}{HTML}{B22B31}
    \definecolor{ansi-green}{HTML}{00A250}
    \definecolor{ansi-green-intense}{HTML}{007427}
    \definecolor{ansi-yellow}{HTML}{DDB62B}
    \definecolor{ansi-yellow-intense}{HTML}{B27D12}
    \definecolor{ansi-blue}{HTML}{208FFB}
    \definecolor{ansi-blue-intense}{HTML}{0065CA}
    \definecolor{ansi-magenta}{HTML}{D160C4}
    \definecolor{ansi-magenta-intense}{HTML}{A03196}
    \definecolor{ansi-cyan}{HTML}{60C6C8}
    \definecolor{ansi-cyan-intense}{HTML}{258F8F}
    \definecolor{ansi-white}{HTML}{C5C1B4}
    \definecolor{ansi-white-intense}{HTML}{A1A6B2}

    % commands and environments needed by pandoc snippets
    % extracted from the output of `pandoc -s`
    \providecommand{\tightlist}{%
      \setlength{\itemsep}{0pt}\setlength{\parskip}{0pt}}
    \DefineVerbatimEnvironment{Highlighting}{Verbatim}{commandchars=\\\{\}}
    % Add ',fontsize=\small' for more characters per line
    \newenvironment{Shaded}{}{}
    \newcommand{\KeywordTok}[1]{\textcolor[rgb]{0.00,0.44,0.13}{\textbf{{#1}}}}
    \newcommand{\DataTypeTok}[1]{\textcolor[rgb]{0.56,0.13,0.00}{{#1}}}
    \newcommand{\DecValTok}[1]{\textcolor[rgb]{0.25,0.63,0.44}{{#1}}}
    \newcommand{\BaseNTok}[1]{\textcolor[rgb]{0.25,0.63,0.44}{{#1}}}
    \newcommand{\FloatTok}[1]{\textcolor[rgb]{0.25,0.63,0.44}{{#1}}}
    \newcommand{\CharTok}[1]{\textcolor[rgb]{0.25,0.44,0.63}{{#1}}}
    \newcommand{\StringTok}[1]{\textcolor[rgb]{0.25,0.44,0.63}{{#1}}}
    \newcommand{\CommentTok}[1]{\textcolor[rgb]{0.38,0.63,0.69}{\textit{{#1}}}}
    \newcommand{\OtherTok}[1]{\textcolor[rgb]{0.00,0.44,0.13}{{#1}}}
    \newcommand{\AlertTok}[1]{\textcolor[rgb]{1.00,0.00,0.00}{\textbf{{#1}}}}
    \newcommand{\FunctionTok}[1]{\textcolor[rgb]{0.02,0.16,0.49}{{#1}}}
    \newcommand{\RegionMarkerTok}[1]{{#1}}
    \newcommand{\ErrorTok}[1]{\textcolor[rgb]{1.00,0.00,0.00}{\textbf{{#1}}}}
    \newcommand{\NormalTok}[1]{{#1}}
    
    % Additional commands for more recent versions of Pandoc
    \newcommand{\ConstantTok}[1]{\textcolor[rgb]{0.53,0.00,0.00}{{#1}}}
    \newcommand{\SpecialCharTok}[1]{\textcolor[rgb]{0.25,0.44,0.63}{{#1}}}
    \newcommand{\VerbatimStringTok}[1]{\textcolor[rgb]{0.25,0.44,0.63}{{#1}}}
    \newcommand{\SpecialStringTok}[1]{\textcolor[rgb]{0.73,0.40,0.53}{{#1}}}
    \newcommand{\ImportTok}[1]{{#1}}
    \newcommand{\DocumentationTok}[1]{\textcolor[rgb]{0.73,0.13,0.13}{\textit{{#1}}}}
    \newcommand{\AnnotationTok}[1]{\textcolor[rgb]{0.38,0.63,0.69}{\textbf{\textit{{#1}}}}}
    \newcommand{\CommentVarTok}[1]{\textcolor[rgb]{0.38,0.63,0.69}{\textbf{\textit{{#1}}}}}
    \newcommand{\VariableTok}[1]{\textcolor[rgb]{0.10,0.09,0.49}{{#1}}}
    \newcommand{\ControlFlowTok}[1]{\textcolor[rgb]{0.00,0.44,0.13}{\textbf{{#1}}}}
    \newcommand{\OperatorTok}[1]{\textcolor[rgb]{0.40,0.40,0.40}{{#1}}}
    \newcommand{\BuiltInTok}[1]{{#1}}
    \newcommand{\ExtensionTok}[1]{{#1}}
    \newcommand{\PreprocessorTok}[1]{\textcolor[rgb]{0.74,0.48,0.00}{{#1}}}
    \newcommand{\AttributeTok}[1]{\textcolor[rgb]{0.49,0.56,0.16}{{#1}}}
    \newcommand{\InformationTok}[1]{\textcolor[rgb]{0.38,0.63,0.69}{\textbf{\textit{{#1}}}}}
    \newcommand{\WarningTok}[1]{\textcolor[rgb]{0.38,0.63,0.69}{\textbf{\textit{{#1}}}}}
    
    
    % Define a nice break command that doesn't care if a line doesn't already
    % exist.
    \def\br{\hspace*{\fill} \\* }
    % Math Jax compatability definitions
    \def\gt{>}
    \def\lt{<}
    % Document parameters
    \title{OOPy-Relations}
    
    
    

    % Pygments definitions
    
\makeatletter
\def\PY@reset{\let\PY@it=\relax \let\PY@bf=\relax%
    \let\PY@ul=\relax \let\PY@tc=\relax%
    \let\PY@bc=\relax \let\PY@ff=\relax}
\def\PY@tok#1{\csname PY@tok@#1\endcsname}
\def\PY@toks#1+{\ifx\relax#1\empty\else%
    \PY@tok{#1}\expandafter\PY@toks\fi}
\def\PY@do#1{\PY@bc{\PY@tc{\PY@ul{%
    \PY@it{\PY@bf{\PY@ff{#1}}}}}}}
\def\PY#1#2{\PY@reset\PY@toks#1+\relax+\PY@do{#2}}

\expandafter\def\csname PY@tok@w\endcsname{\def\PY@tc##1{\textcolor[rgb]{0.73,0.73,0.73}{##1}}}
\expandafter\def\csname PY@tok@c\endcsname{\let\PY@it=\textit\def\PY@tc##1{\textcolor[rgb]{0.25,0.50,0.50}{##1}}}
\expandafter\def\csname PY@tok@cp\endcsname{\def\PY@tc##1{\textcolor[rgb]{0.74,0.48,0.00}{##1}}}
\expandafter\def\csname PY@tok@k\endcsname{\let\PY@bf=\textbf\def\PY@tc##1{\textcolor[rgb]{0.00,0.50,0.00}{##1}}}
\expandafter\def\csname PY@tok@kp\endcsname{\def\PY@tc##1{\textcolor[rgb]{0.00,0.50,0.00}{##1}}}
\expandafter\def\csname PY@tok@kt\endcsname{\def\PY@tc##1{\textcolor[rgb]{0.69,0.00,0.25}{##1}}}
\expandafter\def\csname PY@tok@o\endcsname{\def\PY@tc##1{\textcolor[rgb]{0.40,0.40,0.40}{##1}}}
\expandafter\def\csname PY@tok@ow\endcsname{\let\PY@bf=\textbf\def\PY@tc##1{\textcolor[rgb]{0.67,0.13,1.00}{##1}}}
\expandafter\def\csname PY@tok@nb\endcsname{\def\PY@tc##1{\textcolor[rgb]{0.00,0.50,0.00}{##1}}}
\expandafter\def\csname PY@tok@nf\endcsname{\def\PY@tc##1{\textcolor[rgb]{0.00,0.00,1.00}{##1}}}
\expandafter\def\csname PY@tok@nc\endcsname{\let\PY@bf=\textbf\def\PY@tc##1{\textcolor[rgb]{0.00,0.00,1.00}{##1}}}
\expandafter\def\csname PY@tok@nn\endcsname{\let\PY@bf=\textbf\def\PY@tc##1{\textcolor[rgb]{0.00,0.00,1.00}{##1}}}
\expandafter\def\csname PY@tok@ne\endcsname{\let\PY@bf=\textbf\def\PY@tc##1{\textcolor[rgb]{0.82,0.25,0.23}{##1}}}
\expandafter\def\csname PY@tok@nv\endcsname{\def\PY@tc##1{\textcolor[rgb]{0.10,0.09,0.49}{##1}}}
\expandafter\def\csname PY@tok@no\endcsname{\def\PY@tc##1{\textcolor[rgb]{0.53,0.00,0.00}{##1}}}
\expandafter\def\csname PY@tok@nl\endcsname{\def\PY@tc##1{\textcolor[rgb]{0.63,0.63,0.00}{##1}}}
\expandafter\def\csname PY@tok@ni\endcsname{\let\PY@bf=\textbf\def\PY@tc##1{\textcolor[rgb]{0.60,0.60,0.60}{##1}}}
\expandafter\def\csname PY@tok@na\endcsname{\def\PY@tc##1{\textcolor[rgb]{0.49,0.56,0.16}{##1}}}
\expandafter\def\csname PY@tok@nt\endcsname{\let\PY@bf=\textbf\def\PY@tc##1{\textcolor[rgb]{0.00,0.50,0.00}{##1}}}
\expandafter\def\csname PY@tok@nd\endcsname{\def\PY@tc##1{\textcolor[rgb]{0.67,0.13,1.00}{##1}}}
\expandafter\def\csname PY@tok@s\endcsname{\def\PY@tc##1{\textcolor[rgb]{0.73,0.13,0.13}{##1}}}
\expandafter\def\csname PY@tok@sd\endcsname{\let\PY@it=\textit\def\PY@tc##1{\textcolor[rgb]{0.73,0.13,0.13}{##1}}}
\expandafter\def\csname PY@tok@si\endcsname{\let\PY@bf=\textbf\def\PY@tc##1{\textcolor[rgb]{0.73,0.40,0.53}{##1}}}
\expandafter\def\csname PY@tok@se\endcsname{\let\PY@bf=\textbf\def\PY@tc##1{\textcolor[rgb]{0.73,0.40,0.13}{##1}}}
\expandafter\def\csname PY@tok@sr\endcsname{\def\PY@tc##1{\textcolor[rgb]{0.73,0.40,0.53}{##1}}}
\expandafter\def\csname PY@tok@ss\endcsname{\def\PY@tc##1{\textcolor[rgb]{0.10,0.09,0.49}{##1}}}
\expandafter\def\csname PY@tok@sx\endcsname{\def\PY@tc##1{\textcolor[rgb]{0.00,0.50,0.00}{##1}}}
\expandafter\def\csname PY@tok@m\endcsname{\def\PY@tc##1{\textcolor[rgb]{0.40,0.40,0.40}{##1}}}
\expandafter\def\csname PY@tok@gh\endcsname{\let\PY@bf=\textbf\def\PY@tc##1{\textcolor[rgb]{0.00,0.00,0.50}{##1}}}
\expandafter\def\csname PY@tok@gu\endcsname{\let\PY@bf=\textbf\def\PY@tc##1{\textcolor[rgb]{0.50,0.00,0.50}{##1}}}
\expandafter\def\csname PY@tok@gd\endcsname{\def\PY@tc##1{\textcolor[rgb]{0.63,0.00,0.00}{##1}}}
\expandafter\def\csname PY@tok@gi\endcsname{\def\PY@tc##1{\textcolor[rgb]{0.00,0.63,0.00}{##1}}}
\expandafter\def\csname PY@tok@gr\endcsname{\def\PY@tc##1{\textcolor[rgb]{1.00,0.00,0.00}{##1}}}
\expandafter\def\csname PY@tok@ge\endcsname{\let\PY@it=\textit}
\expandafter\def\csname PY@tok@gs\endcsname{\let\PY@bf=\textbf}
\expandafter\def\csname PY@tok@gp\endcsname{\let\PY@bf=\textbf\def\PY@tc##1{\textcolor[rgb]{0.00,0.00,0.50}{##1}}}
\expandafter\def\csname PY@tok@go\endcsname{\def\PY@tc##1{\textcolor[rgb]{0.53,0.53,0.53}{##1}}}
\expandafter\def\csname PY@tok@gt\endcsname{\def\PY@tc##1{\textcolor[rgb]{0.00,0.27,0.87}{##1}}}
\expandafter\def\csname PY@tok@err\endcsname{\def\PY@bc##1{\setlength{\fboxsep}{0pt}\fcolorbox[rgb]{1.00,0.00,0.00}{1,1,1}{\strut ##1}}}
\expandafter\def\csname PY@tok@kc\endcsname{\let\PY@bf=\textbf\def\PY@tc##1{\textcolor[rgb]{0.00,0.50,0.00}{##1}}}
\expandafter\def\csname PY@tok@kd\endcsname{\let\PY@bf=\textbf\def\PY@tc##1{\textcolor[rgb]{0.00,0.50,0.00}{##1}}}
\expandafter\def\csname PY@tok@kn\endcsname{\let\PY@bf=\textbf\def\PY@tc##1{\textcolor[rgb]{0.00,0.50,0.00}{##1}}}
\expandafter\def\csname PY@tok@kr\endcsname{\let\PY@bf=\textbf\def\PY@tc##1{\textcolor[rgb]{0.00,0.50,0.00}{##1}}}
\expandafter\def\csname PY@tok@bp\endcsname{\def\PY@tc##1{\textcolor[rgb]{0.00,0.50,0.00}{##1}}}
\expandafter\def\csname PY@tok@fm\endcsname{\def\PY@tc##1{\textcolor[rgb]{0.00,0.00,1.00}{##1}}}
\expandafter\def\csname PY@tok@vc\endcsname{\def\PY@tc##1{\textcolor[rgb]{0.10,0.09,0.49}{##1}}}
\expandafter\def\csname PY@tok@vg\endcsname{\def\PY@tc##1{\textcolor[rgb]{0.10,0.09,0.49}{##1}}}
\expandafter\def\csname PY@tok@vi\endcsname{\def\PY@tc##1{\textcolor[rgb]{0.10,0.09,0.49}{##1}}}
\expandafter\def\csname PY@tok@vm\endcsname{\def\PY@tc##1{\textcolor[rgb]{0.10,0.09,0.49}{##1}}}
\expandafter\def\csname PY@tok@sa\endcsname{\def\PY@tc##1{\textcolor[rgb]{0.73,0.13,0.13}{##1}}}
\expandafter\def\csname PY@tok@sb\endcsname{\def\PY@tc##1{\textcolor[rgb]{0.73,0.13,0.13}{##1}}}
\expandafter\def\csname PY@tok@sc\endcsname{\def\PY@tc##1{\textcolor[rgb]{0.73,0.13,0.13}{##1}}}
\expandafter\def\csname PY@tok@dl\endcsname{\def\PY@tc##1{\textcolor[rgb]{0.73,0.13,0.13}{##1}}}
\expandafter\def\csname PY@tok@s2\endcsname{\def\PY@tc##1{\textcolor[rgb]{0.73,0.13,0.13}{##1}}}
\expandafter\def\csname PY@tok@sh\endcsname{\def\PY@tc##1{\textcolor[rgb]{0.73,0.13,0.13}{##1}}}
\expandafter\def\csname PY@tok@s1\endcsname{\def\PY@tc##1{\textcolor[rgb]{0.73,0.13,0.13}{##1}}}
\expandafter\def\csname PY@tok@mb\endcsname{\def\PY@tc##1{\textcolor[rgb]{0.40,0.40,0.40}{##1}}}
\expandafter\def\csname PY@tok@mf\endcsname{\def\PY@tc##1{\textcolor[rgb]{0.40,0.40,0.40}{##1}}}
\expandafter\def\csname PY@tok@mh\endcsname{\def\PY@tc##1{\textcolor[rgb]{0.40,0.40,0.40}{##1}}}
\expandafter\def\csname PY@tok@mi\endcsname{\def\PY@tc##1{\textcolor[rgb]{0.40,0.40,0.40}{##1}}}
\expandafter\def\csname PY@tok@il\endcsname{\def\PY@tc##1{\textcolor[rgb]{0.40,0.40,0.40}{##1}}}
\expandafter\def\csname PY@tok@mo\endcsname{\def\PY@tc##1{\textcolor[rgb]{0.40,0.40,0.40}{##1}}}
\expandafter\def\csname PY@tok@ch\endcsname{\let\PY@it=\textit\def\PY@tc##1{\textcolor[rgb]{0.25,0.50,0.50}{##1}}}
\expandafter\def\csname PY@tok@cm\endcsname{\let\PY@it=\textit\def\PY@tc##1{\textcolor[rgb]{0.25,0.50,0.50}{##1}}}
\expandafter\def\csname PY@tok@cpf\endcsname{\let\PY@it=\textit\def\PY@tc##1{\textcolor[rgb]{0.25,0.50,0.50}{##1}}}
\expandafter\def\csname PY@tok@c1\endcsname{\let\PY@it=\textit\def\PY@tc##1{\textcolor[rgb]{0.25,0.50,0.50}{##1}}}
\expandafter\def\csname PY@tok@cs\endcsname{\let\PY@it=\textit\def\PY@tc##1{\textcolor[rgb]{0.25,0.50,0.50}{##1}}}

\def\PYZbs{\char`\\}
\def\PYZus{\char`\_}
\def\PYZob{\char`\{}
\def\PYZcb{\char`\}}
\def\PYZca{\char`\^}
\def\PYZam{\char`\&}
\def\PYZlt{\char`\<}
\def\PYZgt{\char`\>}
\def\PYZsh{\char`\#}
\def\PYZpc{\char`\%}
\def\PYZdl{\char`\$}
\def\PYZhy{\char`\-}
\def\PYZsq{\char`\'}
\def\PYZdq{\char`\"}
\def\PYZti{\char`\~}
% for compatibility with earlier versions
\def\PYZat{@}
\def\PYZlb{[}
\def\PYZrb{]}
\makeatother


    % Exact colors from NB
    \definecolor{incolor}{rgb}{0.0, 0.0, 0.5}
    \definecolor{outcolor}{rgb}{0.545, 0.0, 0.0}



    
    % Prevent overflowing lines due to hard-to-break entities
    \sloppy 
    % Setup hyperref package
    \hypersetup{
      breaklinks=true,  % so long urls are correctly broken across lines
      colorlinks=true,
      urlcolor=urlcolor,
      linkcolor=linkcolor,
      citecolor=citecolor,
      }
    % Slightly bigger margins than the latex defaults
    
    \geometry{verbose,tmargin=1in,bmargin=1in,lmargin=1in,rmargin=1in}
    
    

    \begin{document}
    
    
    \maketitle
    
    

    
    \section{Object Oriented Modeling:
Relations}\label{object-oriented-modeling-relations}

In this note we focus on classes and relations: How various objects in
an application can collaborate with each other?

    \subsection{Review:}\label{review}

We know: - A class is a descriptor for a set of entities that share the
same attributes, operations, relations and behaviour. It does not have a
life-time. It does not exist in run-time. - An object is an instance of
a class with identity, state and behaviour. It has a life-time. It
consumes memory at run-time.

    \subsection{Motivations}\label{motivations}

Examples:

\begin{itemize}
\item
  In a larger applications we may have a group of objects that are
  \emph{inherently} sharing attributes and behaviours: in a library some
  people are working as employees and authors are writing books. But,
  both employees and authors are persons: they have shared properties.
  How do we model this?
\item
  An author writes a book. A library lends books. The objects involved
  in these type of scenarios are not sharing / \emph{inheriting}
  properties, but they are \emph{collaborating}. How one can specify the
  collaborations?
\end{itemize}

    \subsection{Class Diagram}\label{class-diagram}

A class diagram of a (software) system specifies: - Static model of
(software) units collaborating to each other. - Units are represented as
\emph{Classes}, - Collaborations are denoted statically as
\emph{Relations} between classes.

Here we explain three fundamental relations between classes. We will try
to present how various relations can be implemented in Python.

    \subsubsection{Association}\label{association}

\textbf{Definition}: An association describes \emph{lifelong}
connections / collaborations among objects. Association between two
classes means: An object from one class, during its lifetime, has
reference to object(s) of another class.

\textbf{Example}: Each person has one or more address(es).

\textbf{UML}: The picture here shows how an association between two
classes can be represented in UML. It specifies: an object from Person
has \emph{exactly one} address which is defined as \emph{private}. The
number is called the carinality of the relation, the name specifies the
variable name to be defined as an attribute of Person and '-' specifies
the visibility (private or public) of the address in Person.

To specify an association one can define \emph{navigability} (the
direction of association), \emph{multiplicity} (cardinality of objects
involved in the relation), \emph{visibility} (level of accessibility),
\emph{role} (how associated object is involved in the relation).

In the model above, navigability is from Person to the Address,
visibility is private, multiplicity is exactly one for the address and
the role of Address is defined as address.

As it is defined, an association represents a reference. But, how an
object from a Person has access to objects from Address? Check the code
below.

\textbf{Programming}: In programming, an association is implemented as
an attribute. A Person (an object instantiated from a Person) has access
to an Address (an object instantiated from an Address). The best way is
to define an attribute for the Person of type Address. Then any object
of Person can access to its address whenever it is needed.

    \begin{Verbatim}[commandchars=\\\{\}]
{\color{incolor}In [{\color{incolor}23}]:} \PY{k}{class} \PY{n+nc}{Person}\PY{p}{:}
             \PY{k}{def} \PY{n+nf}{\PYZus{}\PYZus{}init\PYZus{}\PYZus{}}\PY{p}{(}\PY{n+nb+bp}{self}\PY{p}{,}\PY{n}{fn}\PY{p}{,}\PY{n}{ln}\PY{p}{,}\PY{n}{ad}\PY{p}{)}\PY{p}{:}  \PY{c+c1}{\PYZsh{} let\PYZsq{}s initialize some attributes, including the address}
                 \PY{n+nb+bp}{self}\PY{o}{.}\PY{n}{first\PYZus{}name} \PY{o}{=} \PY{n}{fn}
                 \PY{n+nb+bp}{self}\PY{o}{.}\PY{n}{last\PYZus{}name} \PY{o}{=} \PY{n}{ln}
                 \PY{n+nb+bp}{self}\PY{o}{.}\PY{n}{\PYZus{}\PYZus{}address} \PY{o}{=} \PY{n}{ad}    \PY{c+c1}{\PYZsh{} as it is specified in the model, address is private}
         
             \PY{k}{def} \PY{n+nf}{getInfo}\PY{p}{(}\PY{n+nb+bp}{self}\PY{p}{)}\PY{p}{:}      
                 \PY{c+c1}{\PYZsh{} check: how the address object is used}
                 \PY{n}{res} \PY{o}{=} \PY{l+s+s1}{\PYZsq{}}\PY{l+s+s1}{[ Name ]:}\PY{l+s+s1}{\PYZsq{}}\PY{o}{+}\PY{n+nb+bp}{self}\PY{o}{.}\PY{n}{first\PYZus{}name}\PY{o}{+}\PY{l+s+s1}{\PYZsq{}}\PY{l+s+s1}{ }\PY{l+s+s1}{\PYZsq{}}\PY{o}{+}\PY{n+nb+bp}{self}\PY{o}{.}\PY{n}{last\PYZus{}name}\PY{o}{+}\PY{n+nb+bp}{self}\PY{o}{.}\PY{n}{\PYZus{}\PYZus{}address}\PY{o}{.}\PY{n}{getAddress}\PY{p}{(}\PY{p}{)}
                 \PY{k}{return} \PY{n}{res}
         
         \PY{k}{class} \PY{n+nc}{Address}\PY{p}{:}
             \PY{k}{def} \PY{n+nf}{\PYZus{}\PYZus{}init\PYZus{}\PYZus{}}\PY{p}{(}\PY{n+nb+bp}{self}\PY{p}{,}\PY{n}{cnt}\PY{p}{,}\PY{n}{cty}\PY{p}{,}\PY{n}{st}\PY{p}{,}\PY{n}{num}\PY{p}{)}\PY{p}{:}
                 \PY{n+nb+bp}{self}\PY{o}{.}\PY{n}{country} \PY{o}{=} \PY{n}{cnt}
                 \PY{n+nb+bp}{self}\PY{o}{.}\PY{n}{city} \PY{o}{=} \PY{n}{cty}
                 \PY{n+nb+bp}{self}\PY{o}{.}\PY{n}{street} \PY{o}{=} \PY{n}{st}
                 \PY{n+nb+bp}{self}\PY{o}{.}\PY{n}{number} \PY{o}{=} \PY{n}{num}
                 \PY{n+nb+bp}{self}\PY{o}{.}\PY{n}{\PYZus{}\PYZus{}sep} \PY{o}{=} \PY{l+s+s1}{\PYZsq{}}\PY{l+s+s1}{ , }\PY{l+s+s1}{\PYZsq{}}
         
             \PY{k}{def} \PY{n+nf}{getAddress}\PY{p}{(}\PY{n+nb+bp}{self}\PY{p}{)}\PY{p}{:}
                 \PY{k}{return} \PY{l+s+s1}{\PYZsq{}}\PY{l+s+s1}{[ Address ]:}\PY{l+s+s1}{\PYZsq{}}\PY{o}{+}\PY{n+nb+bp}{self}\PY{o}{.}\PY{n}{country}\PY{o}{+}\PY{n+nb+bp}{self}\PY{o}{.}\PY{n}{\PYZus{}\PYZus{}sep}\PY{o}{+}\PY{n+nb+bp}{self}\PY{o}{.}\PY{n}{city}\PY{o}{+}\PY{n+nb+bp}{self}\PY{o}{.}\PY{n}{\PYZus{}\PYZus{}sep}\PY{o}{+}\PY{n+nb+bp}{self}\PY{o}{.}\PY{n}{street}\PY{o}{+}\PY{n+nb}{str}\PY{p}{(}\PY{n+nb+bp}{self}\PY{o}{.}\PY{n}{number}\PY{p}{)}
         
         \PY{k}{if} \PY{n+nv+vm}{\PYZus{}\PYZus{}name\PYZus{}\PYZus{}} \PY{o}{==} \PY{l+s+s1}{\PYZsq{}}\PY{l+s+s1}{\PYZus{}\PYZus{}main\PYZus{}\PYZus{}}\PY{l+s+s1}{\PYZsq{}}\PY{p}{:}
             \PY{n}{pa} \PY{o}{=} \PY{n}{Address}\PY{p}{(}\PY{l+s+s1}{\PYZsq{}}\PY{l+s+s1}{The Netherlands}\PY{l+s+s1}{\PYZsq{}}\PY{p}{,}\PY{l+s+s1}{\PYZsq{}}\PY{l+s+s1}{Rotterdam}\PY{l+s+s1}{\PYZsq{}}\PY{p}{,}\PY{l+s+s1}{\PYZsq{}}\PY{l+s+s1}{Rembrandt}\PY{l+s+s1}{\PYZsq{}}\PY{p}{,}\PY{l+m+mi}{4}\PY{p}{)}
             \PY{n}{p} \PY{o}{=} \PY{n}{Person}\PY{p}{(}\PY{l+s+s1}{\PYZsq{}}\PY{l+s+s1}{Dianna}\PY{l+s+s1}{\PYZsq{}}\PY{p}{,}\PY{l+s+s1}{\PYZsq{}}\PY{l+s+s1}{King}\PY{l+s+s1}{\PYZsq{}}\PY{p}{,}\PY{n}{pa}\PY{p}{)}  \PY{c+c1}{\PYZsh{} check: how we pass the address}
         
             \PY{n+nb}{print}\PY{p}{(}\PY{n}{p}\PY{o}{.}\PY{n}{getInfo}\PY{p}{(}\PY{p}{)}\PY{p}{)}
\end{Verbatim}


    \begin{Verbatim}[commandchars=\\\{\}]
[ Name ]:Dianna King[ Address ]:The Netherlands , Rotterdam , Rembrandt4

    \end{Verbatim}

    \textbf{Exercise}: Intrepret the model below. It specifies that each
person can have one or more addresses. How would you implement it?
Update your code to meet the UML specification. What are the cardinality
and visibility in this model?

    \subsubsection{Dependency}\label{dependency}

\textbf{Definition of Dependency}: Class A has \emph{dependecy} relation
with B when an object from A has access to an instance of B
\emph{temporarily} during its life-time.

\textbf{Example}: In a board game, a player throws a dice to move the
pawns. The player does not need to keep the reference to the dice during
the whole play. The player gets the dice, thorws the dice, uses the
result and release the access to the dice.

\textbf{UML}: The digram below shows how one can model the depenedency
relation between a player and a dice. Note that the rest of the model
and application is not presented here.

\textbf{Programming}: The code below presents how one can implement the
concept.

    \begin{Verbatim}[commandchars=\\\{\}]
{\color{incolor}In [{\color{incolor}24}]:} \PY{k+kn}{from} \PY{n+nn}{random} \PY{k}{import} \PY{o}{*}
         
         \PY{k}{class} \PY{n+nc}{Dice}\PY{p}{:}
             \PY{k}{def} \PY{n+nf}{\PYZus{}\PYZus{}init\PYZus{}\PYZus{}}\PY{p}{(}\PY{n+nb+bp}{self}\PY{p}{,}\PY{n}{s}\PY{o}{=}\PY{l+m+mi}{6}\PY{p}{)}\PY{p}{:}  \PY{c+c1}{\PYZsh{} default is a normal dice}
                 \PY{n+nb+bp}{self}\PY{o}{.}\PY{n}{sides} \PY{o}{=} \PY{n}{s}         \PY{c+c1}{\PYZsh{} max of sides}
                 \PY{n+nb+bp}{self}\PY{o}{.}\PY{n}{state} \PY{o}{=} \PY{n}{randint}\PY{p}{(}\PY{l+m+mi}{1}\PY{p}{,}\PY{n+nb+bp}{self}\PY{o}{.}\PY{n}{sides}\PY{p}{)}  \PY{c+c1}{\PYZsh{} the state of the dice, a random number}
         
             \PY{k}{def} \PY{n+nf}{throw}\PY{p}{(}\PY{n+nb+bp}{self}\PY{p}{)}\PY{p}{:}
                 \PY{n+nb+bp}{self}\PY{o}{.}\PY{n}{state} \PY{o}{=} \PY{n}{randint}\PY{p}{(}\PY{l+m+mi}{1}\PY{p}{,}\PY{n+nb+bp}{self}\PY{o}{.}\PY{n}{sides}\PY{p}{)}
                 \PY{k}{return} \PY{n+nb+bp}{self}\PY{o}{.}\PY{n}{state}
         
         \PY{k}{class} \PY{n+nc}{Player}\PY{p}{:}
             \PY{k}{def} \PY{n+nf}{\PYZus{}\PYZus{}init\PYZus{}\PYZus{}}\PY{p}{(}\PY{n+nb+bp}{self}\PY{p}{,}\PY{n}{n}\PY{o}{=}\PY{l+s+s1}{\PYZsq{}}\PY{l+s+s1}{\PYZsq{}}\PY{p}{)}\PY{p}{:}
                 \PY{n+nb+bp}{self}\PY{o}{.}\PY{n}{name} \PY{o}{=} \PY{n}{n}
         
             \PY{k}{def} \PY{n+nf}{play}\PY{p}{(}\PY{n+nb+bp}{self}\PY{p}{,}\PY{n}{d}\PY{p}{)}\PY{p}{:}  \PY{c+c1}{\PYZsh{} check: the scope of using the dice is this method}
                 \PY{n}{x} \PY{o}{=} \PY{n}{d}\PY{o}{.}\PY{n}{throw}\PY{p}{(}\PY{p}{)}
                 \PY{n+nb+bp}{self}\PY{o}{.}\PY{n}{move}\PY{p}{(}\PY{n}{x}\PY{p}{)}
                 \PY{c+c1}{\PYZsh{} rest of the code can be here ...}
         
             \PY{k}{def} \PY{n+nf}{move}\PY{p}{(}\PY{n+nb+bp}{self}\PY{p}{,}\PY{n}{n}\PY{p}{)}\PY{p}{:}  \PY{c+c1}{\PYZsh{} check: this method does not depend on Dice}
                 \PY{c+c1}{\PYZsh{} move your pawns here ...}
                 \PY{n+nb}{print}\PY{p}{(}\PY{n+nb+bp}{self}\PY{o}{.}\PY{n}{name}\PY{p}{,}\PY{l+s+s1}{\PYZsq{}}\PY{l+s+s1}{ moved pawns }\PY{l+s+s1}{\PYZsq{}}\PY{p}{,}\PY{n}{n}\PY{p}{)}
         
         \PY{k}{if} \PY{n+nv+vm}{\PYZus{}\PYZus{}name\PYZus{}\PYZus{}} \PY{o}{==} \PY{l+s+s1}{\PYZsq{}}\PY{l+s+s1}{\PYZus{}\PYZus{}main\PYZus{}\PYZus{}}\PY{l+s+s1}{\PYZsq{}}\PY{p}{:}
             \PY{n}{d} \PY{o}{=} \PY{n}{Dice}\PY{p}{(}\PY{p}{)}  \PY{c+c1}{\PYZsh{} let\PYZsq{}s create a normal dice}
             \PY{n}{p} \PY{o}{=} \PY{n}{Player}\PY{p}{(}\PY{l+s+s1}{\PYZsq{}}\PY{l+s+s1}{Alex}\PY{l+s+s1}{\PYZsq{}}\PY{p}{)}  \PY{c+c1}{\PYZsh{} let\PYZsq{}s instantiate a player}
         
             \PY{n}{p}\PY{o}{.}\PY{n}{play}\PY{p}{(}\PY{n}{d}\PY{p}{)}  \PY{c+c1}{\PYZsh{} we ask the player to play the game}
             \PY{c+c1}{\PYZsh{} the rest of the code}
\end{Verbatim}


    \begin{Verbatim}[commandchars=\\\{\}]
Alex  moved pawns  6

    \end{Verbatim}

    \subsubsection{Inheritance
(Generalization)}\label{inheritance-generalization}

\textbf{Definition of Inheritance}: Inheritance (or generalization)
specifies a hierarchy of abstractions, in which a subclass (child)
inherits structure or behaviour from a superclass (parent). In this
relationship a subclass extends features of its superclass.

\textbf{Example}: Student is a kind of Person. A Student can inherit
some features from Person. Moreover, a Student can extend the features
of a Person. Student is a subclass (or child) and Person is called a
superclass (or parent).

\textbf{UML}: The picture here shows how an inheritance between two
classes can be represented in UML.

\textbf{Programming}: We have practiced how to implement classes. How
can we specify that a Student is a child of Person? The following
implementation presents how one can implement that Student inherits from
Person. Moreover, it shows how shared attribtes (first name, last name)
and shared behaviour (getInfo()) are defined in the parent class. Then,
the child class, i.e. Student can add more attrbutes to the parent and
can also extend the behaviour. The keyword \textbf{super()} allows the
child class to access the attributes, methods of the parent class.

    \begin{Verbatim}[commandchars=\\\{\}]
{\color{incolor}In [{\color{incolor}25}]:} \PY{k}{class} \PY{n+nc}{Person}\PY{p}{:}
             \PY{k}{def} \PY{n+nf}{\PYZus{}\PYZus{}init\PYZus{}\PYZus{}}\PY{p}{(}\PY{n+nb+bp}{self}\PY{p}{,}\PY{n}{fn}\PY{p}{,}\PY{n}{ln}\PY{p}{)}\PY{p}{:}  \PY{c+c1}{\PYZsh{} let\PYZsq{}s initialize some attributes}
                 \PY{n+nb+bp}{self}\PY{o}{.}\PY{n}{first\PYZus{}name} \PY{o}{=} \PY{n}{fn}
                 \PY{n+nb+bp}{self}\PY{o}{.}\PY{n}{last\PYZus{}name} \PY{o}{=} \PY{n}{ln}
         
             \PY{k}{def} \PY{n+nf}{getInfo}\PY{p}{(}\PY{n+nb+bp}{self}\PY{p}{)}\PY{p}{:}
                 \PY{n}{res} \PY{o}{=} \PY{l+s+s2}{\PYZdq{}}\PY{l+s+s2}{Name : }\PY{l+s+s2}{\PYZdq{}}\PY{o}{+}\PY{n+nb+bp}{self}\PY{o}{.}\PY{n}{first\PYZus{}name}\PY{o}{+}\PY{l+s+s1}{\PYZsq{}}\PY{l+s+s1}{ }\PY{l+s+s1}{\PYZsq{}}\PY{o}{+}\PY{n+nb+bp}{self}\PY{o}{.}\PY{n}{last\PYZus{}name}
                 \PY{k}{return} \PY{n}{res}
         
         
         \PY{k}{class} \PY{n+nc}{Student}\PY{p}{(}\PY{n}{Person}\PY{p}{)}\PY{p}{:}  \PY{c+c1}{\PYZsh{} See how you can specify the parent of a class}
             \PY{k}{def} \PY{n+nf}{\PYZus{}\PYZus{}init\PYZus{}\PYZus{}}\PY{p}{(}\PY{n+nb+bp}{self}\PY{p}{,}\PY{n}{fn}\PY{p}{,}\PY{n}{ln}\PY{p}{,}\PY{n}{sn}\PY{p}{)}\PY{p}{:}
                 \PY{n+nb}{super}\PY{p}{(}\PY{p}{)}\PY{o}{.}\PY{n+nf+fm}{\PYZus{}\PYZus{}init\PYZus{}\PYZus{}}\PY{p}{(}\PY{n}{fn}\PY{p}{,}\PY{n}{ln}\PY{p}{)}   \PY{c+c1}{\PYZsh{} We can call parent\PYZsq{}s initializer : super() refers to the parent class}
                 \PY{n+nb+bp}{self}\PY{o}{.}\PY{n}{student\PYZus{}number} \PY{o}{=} \PY{n}{sn}  \PY{c+c1}{\PYZsh{} Students extends its parent here}
         
             \PY{k}{def} \PY{n+nf}{getInfo}\PY{p}{(}\PY{n+nb+bp}{self}\PY{p}{)}\PY{p}{:}
                 \PY{n}{res} \PY{o}{=} \PY{l+s+s1}{\PYZsq{}}\PY{l+s+s1}{[ Student ]:}\PY{l+s+s1}{\PYZsq{}}\PY{o}{+}\PY{n+nb}{super}\PY{p}{(}\PY{p}{)}\PY{o}{.}\PY{n}{getInfo}\PY{p}{(}\PY{p}{)}\PY{o}{+}\PY{l+s+s1}{\PYZsq{}}\PY{l+s+s1}{ ; }\PY{l+s+s1}{\PYZsq{}}\PY{o}{+}\PY{n+nb+bp}{self}\PY{o}{.}\PY{n}{student\PYZus{}number}  \PY{c+c1}{\PYZsh{} Let\PYZsq{}s extend parent\PYZsq{}s behaviour}
                 \PY{k}{return} \PY{n}{res}
         
         
         \PY{k}{if} \PY{n+nv+vm}{\PYZus{}\PYZus{}name\PYZus{}\PYZus{}}\PY{o}{==}\PY{l+s+s2}{\PYZdq{}}\PY{l+s+s2}{\PYZus{}\PYZus{}main\PYZus{}\PYZus{}}\PY{l+s+s2}{\PYZdq{}}\PY{p}{:}
             
             \PY{n}{p} \PY{o}{=} \PY{n}{Person}\PY{p}{(}\PY{l+s+s1}{\PYZsq{}}\PY{l+s+s1}{Peter}\PY{l+s+s1}{\PYZsq{}}\PY{p}{,}\PY{l+s+s1}{\PYZsq{}}\PY{l+s+s1}{Bell}\PY{l+s+s1}{\PYZsq{}}\PY{p}{)} \PY{c+c1}{\PYZsh{} Check: an object from the parent}
             \PY{n}{std} \PY{o}{=} \PY{n}{Student}\PY{p}{(}\PY{l+s+s1}{\PYZsq{}}\PY{l+s+s1}{Dianna}\PY{l+s+s1}{\PYZsq{}}\PY{p}{,}\PY{l+s+s1}{\PYZsq{}}\PY{l+s+s1}{King}\PY{l+s+s1}{\PYZsq{}}\PY{p}{,}\PY{l+s+s1}{\PYZsq{}}\PY{l+s+s1}{09875673}\PY{l+s+s1}{\PYZsq{}}\PY{p}{)}  \PY{c+c1}{\PYZsh{} Check: an object from the student}
         
             \PY{c+c1}{\PYZsh{} see how same method behaves differently in two objects}
             \PY{n+nb}{print}\PY{p}{(}\PY{l+s+s1}{\PYZsq{}}\PY{l+s+s1}{[ Check ]:}\PY{l+s+s1}{\PYZsq{}}\PY{p}{,} \PY{n}{p}\PY{o}{.}\PY{n}{getInfo}\PY{p}{(}\PY{p}{)}\PY{p}{)}  
             \PY{n+nb}{print}\PY{p}{(}\PY{l+s+s1}{\PYZsq{}}\PY{l+s+s1}{[ Check ]:}\PY{l+s+s1}{\PYZsq{}}\PY{p}{,} \PY{n}{std}\PY{o}{.}\PY{n}{getInfo}\PY{p}{(}\PY{p}{)}\PY{p}{)}  
\end{Verbatim}


    \begin{Verbatim}[commandchars=\\\{\}]
[ Check ]: Name : Peter Bell
[ Check ]: [ Student ]:Name : Dianna King ; 09875673

    \end{Verbatim}

    \textbf{Example}: The picture here shows a Student can be extended
further. See the coresponding implementation below. Check all the
details of inheritance carefully.

    \begin{Verbatim}[commandchars=\\\{\}]
{\color{incolor}In [{\color{incolor}26}]:} \PY{k}{class} \PY{n+nc}{Person}\PY{p}{:}
             \PY{k}{def} \PY{n+nf}{\PYZus{}\PYZus{}init\PYZus{}\PYZus{}}\PY{p}{(}\PY{n+nb+bp}{self}\PY{p}{,}\PY{n}{fn}\PY{p}{,}\PY{n}{ln}\PY{p}{)}\PY{p}{:}  \PY{c+c1}{\PYZsh{} let\PYZsq{}s initialize some attributes}
                 \PY{n+nb+bp}{self}\PY{o}{.}\PY{n}{first\PYZus{}name} \PY{o}{=} \PY{n}{fn}
                 \PY{n+nb+bp}{self}\PY{o}{.}\PY{n}{last\PYZus{}name} \PY{o}{=} \PY{n}{ln}
         
             \PY{k}{def} \PY{n+nf}{getInfo}\PY{p}{(}\PY{n+nb+bp}{self}\PY{p}{)}\PY{p}{:}
                 \PY{n}{res} \PY{o}{=} \PY{l+s+s2}{\PYZdq{}}\PY{l+s+s2}{Name : }\PY{l+s+s2}{\PYZdq{}}\PY{o}{+}\PY{n+nb+bp}{self}\PY{o}{.}\PY{n}{first\PYZus{}name}\PY{o}{+}\PY{l+s+s1}{\PYZsq{}}\PY{l+s+s1}{ }\PY{l+s+s1}{\PYZsq{}}\PY{o}{+}\PY{n+nb+bp}{self}\PY{o}{.}\PY{n}{last\PYZus{}name}
                 \PY{k}{return} \PY{n}{res}
         
         
         \PY{k}{class} \PY{n+nc}{Student}\PY{p}{(}\PY{n}{Person}\PY{p}{)}\PY{p}{:}  \PY{c+c1}{\PYZsh{} See how you can specify the parent of a class}
             \PY{k}{def} \PY{n+nf}{\PYZus{}\PYZus{}init\PYZus{}\PYZus{}}\PY{p}{(}\PY{n+nb+bp}{self}\PY{p}{,}\PY{n}{fn}\PY{p}{,}\PY{n}{ln}\PY{p}{,}\PY{n}{sn}\PY{p}{)}\PY{p}{:}
                 \PY{n+nb}{super}\PY{p}{(}\PY{p}{)}\PY{o}{.}\PY{n+nf+fm}{\PYZus{}\PYZus{}init\PYZus{}\PYZus{}}\PY{p}{(}\PY{n}{fn}\PY{p}{,}\PY{n}{ln}\PY{p}{)}   \PY{c+c1}{\PYZsh{} We can call parent\PYZsq{}s initializer : super() refers to the parent class}
                 \PY{n+nb+bp}{self}\PY{o}{.}\PY{n}{student\PYZus{}number} \PY{o}{=} \PY{n}{sn}  \PY{c+c1}{\PYZsh{} Students extends its parent here}
         
             \PY{k}{def} \PY{n+nf}{getInfo}\PY{p}{(}\PY{n+nb+bp}{self}\PY{p}{)}\PY{p}{:}
                 \PY{n}{res} \PY{o}{=} \PY{l+s+s1}{\PYZsq{}}\PY{l+s+s1}{[ Student ]:}\PY{l+s+s1}{\PYZsq{}}\PY{o}{+}\PY{n+nb}{super}\PY{p}{(}\PY{p}{)}\PY{o}{.}\PY{n}{getInfo}\PY{p}{(}\PY{p}{)}\PY{o}{+}\PY{l+s+s1}{\PYZsq{}}\PY{l+s+s1}{ ; }\PY{l+s+s1}{\PYZsq{}}\PY{o}{+}\PY{n+nb+bp}{self}\PY{o}{.}\PY{n}{student\PYZus{}number}  \PY{c+c1}{\PYZsh{} Let\PYZsq{}s extend parent\PYZsq{}s behaviour}
                 \PY{k}{return} \PY{n}{res}
         
         \PY{k}{class} \PY{n+nc}{BachelorStudent}\PY{p}{(}\PY{n}{Student}\PY{p}{)}\PY{p}{:}
             \PY{k}{def} \PY{n+nf}{\PYZus{}\PYZus{}init\PYZus{}\PYZus{}}\PY{p}{(}\PY{n+nb+bp}{self}\PY{p}{,}\PY{n}{fn}\PY{p}{,}\PY{n}{ln}\PY{p}{,}\PY{n}{sn}\PY{p}{)}\PY{p}{:}
                 \PY{n+nb}{super}\PY{p}{(}\PY{p}{)}\PY{o}{.}\PY{n+nf+fm}{\PYZus{}\PYZus{}init\PYZus{}\PYZus{}}\PY{p}{(}\PY{n}{fn}\PY{p}{,}\PY{n}{ln}\PY{p}{,}\PY{n}{sn}\PY{p}{)}
                 \PY{n+nb+bp}{self}\PY{o}{.}\PY{n}{\PYZus{}\PYZus{}study\PYZus{}duration} \PY{o}{=} \PY{l+m+mi}{4}
         
             \PY{k}{def} \PY{n+nf}{getInfo}\PY{p}{(}\PY{n+nb+bp}{self}\PY{p}{)}\PY{p}{:}
                 \PY{k}{return} \PY{n+nb}{super}\PY{p}{(}\PY{p}{)}\PY{o}{.}\PY{n}{getInfo}\PY{p}{(}\PY{p}{)}\PY{o}{+}\PY{l+s+s1}{\PYZsq{}}\PY{l+s+s1}{ Study Duration is:}\PY{l+s+s1}{\PYZsq{}}\PY{o}{+}\PY{n+nb}{str}\PY{p}{(}\PY{n+nb+bp}{self}\PY{o}{.}\PY{n}{\PYZus{}\PYZus{}study\PYZus{}duration}\PY{p}{)} \PY{c+c1}{\PYZsh{} Let\PYZsq{}s extend parent\PYZsq{}s behaviour}
         
         
         \PY{k}{class} \PY{n+nc}{MasterStudent}\PY{p}{(}\PY{n}{Student}\PY{p}{)}\PY{p}{:}
             \PY{k}{def} \PY{n+nf}{\PYZus{}\PYZus{}init\PYZus{}\PYZus{}}\PY{p}{(}\PY{n+nb+bp}{self}\PY{p}{,}\PY{n}{fn}\PY{p}{,}\PY{n}{ln}\PY{p}{,}\PY{n}{sn}\PY{p}{)}\PY{p}{:}
                 \PY{n+nb}{super}\PY{p}{(}\PY{p}{)}\PY{o}{.}\PY{n+nf+fm}{\PYZus{}\PYZus{}init\PYZus{}\PYZus{}}\PY{p}{(}\PY{n}{fn}\PY{p}{,}\PY{n}{ln}\PY{p}{,}\PY{n}{sn}\PY{p}{)}
                 \PY{n+nb+bp}{self}\PY{o}{.}\PY{n}{\PYZus{}\PYZus{}study\PYZus{}duration} \PY{o}{=} \PY{l+m+mi}{2}
         
             \PY{k}{def} \PY{n+nf}{getInfo}\PY{p}{(}\PY{n+nb+bp}{self}\PY{p}{)}\PY{p}{:}
                 \PY{k}{return} \PY{n+nb}{super}\PY{p}{(}\PY{p}{)}\PY{o}{.}\PY{n}{getInfo}\PY{p}{(}\PY{p}{)}\PY{o}{+}\PY{l+s+s1}{\PYZsq{}}\PY{l+s+s1}{ Study Duration is:}\PY{l+s+s1}{\PYZsq{}}\PY{o}{+}\PY{n+nb}{str}\PY{p}{(}\PY{n+nb+bp}{self}\PY{o}{.}\PY{n}{\PYZus{}\PYZus{}study\PYZus{}duration}\PY{p}{)} \PY{c+c1}{\PYZsh{} Let\PYZsq{}s extend parent\PYZsq{}s behaviour}
         
         
         \PY{k}{if} \PY{n+nv+vm}{\PYZus{}\PYZus{}name\PYZus{}\PYZus{}}\PY{o}{==}\PY{l+s+s2}{\PYZdq{}}\PY{l+s+s2}{\PYZus{}\PYZus{}main\PYZus{}\PYZus{}}\PY{l+s+s2}{\PYZdq{}}\PY{p}{:}
             \PY{n}{std1} \PY{o}{=} \PY{n}{BachelorStudent}\PY{p}{(}\PY{l+s+s1}{\PYZsq{}}\PY{l+s+s1}{Dianna}\PY{l+s+s1}{\PYZsq{}}\PY{p}{,}\PY{l+s+s1}{\PYZsq{}}\PY{l+s+s1}{King}\PY{l+s+s1}{\PYZsq{}}\PY{p}{,}\PY{l+s+s1}{\PYZsq{}}\PY{l+s+s1}{09875673}\PY{l+s+s1}{\PYZsq{}}\PY{p}{)}
             \PY{n}{std2} \PY{o}{=} \PY{n}{MasterStudent}\PY{p}{(}\PY{l+s+s1}{\PYZsq{}}\PY{l+s+s1}{Emma}\PY{l+s+s1}{\PYZsq{}}\PY{p}{,}\PY{l+s+s1}{\PYZsq{}}\PY{l+s+s1}{Lee}\PY{l+s+s1}{\PYZsq{}}\PY{p}{,}\PY{l+s+s1}{\PYZsq{}}\PY{l+s+s1}{09875345}\PY{l+s+s1}{\PYZsq{}}\PY{p}{)}
         
             \PY{c+c1}{\PYZsh{} Check: check how different pieces of the information is accessible through the hierarchy definition}
             \PY{n+nb}{print}\PY{p}{(}\PY{l+s+s1}{\PYZsq{}}\PY{l+s+s1}{[ Check ]:}\PY{l+s+s1}{\PYZsq{}}\PY{p}{,} \PY{n}{std1}\PY{o}{.}\PY{n}{getInfo}\PY{p}{(}\PY{p}{)}\PY{p}{)}
             \PY{n+nb}{print}\PY{p}{(}\PY{l+s+s1}{\PYZsq{}}\PY{l+s+s1}{[ Check ]:}\PY{l+s+s1}{\PYZsq{}}\PY{p}{,} \PY{n}{std2}\PY{o}{.}\PY{n}{getInfo}\PY{p}{(}\PY{p}{)}\PY{p}{)}
\end{Verbatim}


    \begin{Verbatim}[commandchars=\\\{\}]
[ Check ]: [ Student ]:Name : Dianna King ; 09875673 Study Duration is:4
[ Check ]: [ Student ]:Name : Emma Lee ; 09875345 Study Duration is:2

    \end{Verbatim}

    \paragraph{Multiple Inheritance:}\label{multiple-inheritance}

Will come soon ...

    \subsection{Summary}\label{summary}

In this note we have learned: - How one can model and implement
\emph{lifelong} collaborations between objects: Association. - How one
can model and implement \emph{hirerachy} of classes: Inheritance. - How
one can model and implement \emph{temporary} collaborations between
objects: Dependency.

    \subsection{Practice}\label{practice}

\textbf{Exercise}: A teacher identified with an employee number is also
a Person. Moreover, we have different categories of teachers: University
and Hogeschool. Model this hirerachy in UML. Implement corresponding
code. Show how children in each level extends the attributes and
behaviour of the parents.

** Exercise: ** Model (in UML) and implement the following problem
statement: We have two types of vehicles: fueled and unfueled. A bicycle
is kind of unfueled vehicle. Single-fueled and Alternative-fueled
vehicles are in the category of fueled vehicle. Define some proper
attributes and methods for your classes. How do you define a Car in your
design?

** Exercise: ** Model (in UML) and implement the following problem
statement: A Woman and a Man are of type of Person identified with date
of birth of type Date, first and last names. A man can marry to zero or
one woman. A woman can marry to zero or one man.''

** Exercise: ** A game consists of a board with some cells, two die and
several pawns. Two players can play the game. Define classes, model them
in UML. Each player throws two die and move some pawns on the board.
Design proper attributes, methods and implement some of the methods. For
example, implement "the player throws the die and based on the numbers,
move two pawns". Hint: the purpose of this exercise is not to build the
whole game. Focus on designing properties, behaviours and how different
objects can collaborate.


    % Add a bibliography block to the postdoc
    
    
    
    \end{document}
